\documentclass[twocolumn]{article}
\usepackage[margin=0.75cm]{geometry}
\usepackage{hyperref}
\hypersetup{
    colorlinks,
    citecolor=blue,
    filecolor=blue,
    linkcolor=blue,
    urlcolor=blue
}

\usepackage{graphicx, caption, multirow, mathtools, amsfonts, booktabs, siunitx}
\setlength{\columnseprule}{.75pt}
\def\columnseprulecolor{\color{black}}
\newcommand{\overbar}[1]{\mkern 1.5mu\overline{\mkern-1.5mu#1\mkern-1.5mu}\mkern 1.5mu}

\usepackage{tikz}
\usetikzlibrary{positioning, arrows, fit, decorations.pathmorphing}

\setlength{\parindent}{0pt}
\setlength{\parskip}{6pt}

\everymath{\displaystyle}

\title{
	\vspace{-2em}
	\normalsize \textbf{Population Health (MDCN 340) Formula Sheet} \\
	\small Eddie Guo \\
	\dotfill
	\vspace{-5em}
}
\date{}

\begin{document}
\maketitle

\small

\textbf{Mortality}

MR = mortality rate

$\text{crude MR} = \frac{\text{no. of deaths/yr}}{\text{mid-yr pop}}$

$\text{infant MR} = \frac{\text{no. of deaths} <1 \text{ y/o}}{\text{live births/yr}}$

$\text{neonatal MR} = \frac{\text{no. of deaths} <28 \text{ days/o}}{\text{live births/yr}}$

$\text{maternal MR} = \frac{\text{no. of deaths in pregnancy and childbirth}}{\text{live births/yr}}$

$\text{case fatality rate} = \frac{\text{no. of deaths due to disease}}{\text{no. cases of that disease}}$

\dotfill

\textbf{Morbidity}

$\text{point prevalence} = \frac{\text{no. cases at particular time}}{\text{pop}}$

$\text{period prevalence} = \frac{\text{no. cases over interval of time}}{\text{pop at mid-pnt of interval}}$

$\text{incidence risk (cum. incidence)} = \frac{\text{no. new cases over interval of time}}{\text{pop at risk at start of interval}}$

$\text{incidence rate (density)} = \frac{\text{no. new cases over interval of time}}{\text{person-yrs at risk over interval}}$

$\text{absolute risk (AR)} = \frac{\text{no. deaths}}{\text{pop}}$

$\text{relative risk} = \frac{\text{incidence of disease in exposed}}{\text{incidence of disease in unexposed}}$ \hfill OR

$\text{relative risk} = \frac{\text{AR in exposed group}}{\text{AR in unexposed group}}$

\dotfill

\textbf{Disease Burden}

\textit{Acronyms}

PYLL = potential years of life lost

PYLD = potential years of life disabled

DALY = disability adjusted life years

QALY = quality of life adjusted life years

\vspace{-.5em}
\dotfill

\textit{Equations}

$\text{PYLL} = \sum_{\text{age}} \text{(no. deaths at each age)} \text{(life exp at each age)}$

$\text{PYLD} = \text{(no. incident cases)} \text{(avg duration)} \text{(weight)}$

$\quad$0 = perfect health, 1 = death

$\text{DALY} = \text{PYLL} + \text{PYLD}$

\vspace{-.5em}
\dotfill

\textbf{Measuring Healthy Populations}

Direct standardization: modifies standard pop w/ measured pop

Indirect standardization: modfifies measured pop w/ standard pop

$\quad \text{SMR} = \frac{\text{no. of observed deaths}}{\text{no. of expected deaths}}$

$\quad$SMR = standardized mortality ratio


\newpage


\textbf{Confusion Matrix}

\textit{Acronyms}

PPV = positive predictive value

NPV = negative predictive value

\vspace{-.5em}
\dotfill

\textit{Equations}

\begin{table}[h]
    \centering
    \begin{tabular}{lcc}
        \toprule
        & Disease & No disease \\
        \midrule
        Test + & TP & FP \\
        Test - & FN & TN \\
        \bottomrule
    \end{tabular}
\end{table}

$\text{PPV} = \frac{\text{TP}}{\text{TP} + \text{FP}}$ \hfill $\text{NPV} = \frac{\text{TN}}{\text{TN} + \text{FN}}$

$\text{sensitivity} = \frac{\text{TP}}{\text{TP} + \text{FN}}$ \hfill \hfill $\text{specificity} = \frac{\text{TN}}{\text{TN} + \text{FP}}$

\dotfill






\end{document}
