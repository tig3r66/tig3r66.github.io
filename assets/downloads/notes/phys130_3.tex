\documentclass[twocolumn]{article}
\usepackage[margin=0.75cm]{geometry}
\usepackage{hyperref}
\hypersetup{
    colorlinks,
    citecolor=blue,
    filecolor=blue,
    linkcolor=blue,
    urlcolor=blue
}

\usepackage{graphicx, mathtools, multicol, pgfplots, wrapfig, caption, subcaption}
\setlength{\columnseprule}{.75pt}
\def\columnseprulecolor{\color{black}}

\setlength{\parindent}{0pt}
\setlength{\parskip}{6pt}


\begin{document}

\small

\textbf{Doppler} \hspace{8em} $f_0 = f_s \frac{c \pm v_0}{c \mp v_s}$

\begin{multicols}{2}
    \textit{Observer} \vspace{-1em}
    \begin{itemize}
        \item $+v_0$ = towards source
        \item $-v_0$ = away from source
    \end{itemize}

    \textit{Source} \vspace{-1em}
    \begin{itemize}
        \item $-v_s$ = towards observer
        \item $+v_s$ = away from observer
    \end{itemize}
\end{multicols} \vspace{-1.5em}

\dotfill

\textbf{Standing Waves}

General form: $\psi(x, t) = 2A \sin(kx) \sin(\omega t)$

Both ends open/closed: $\lambda = \frac{2l}{n},\ f = \frac{nc}{2l}$

1 open, 1 closed: $\lambda = \frac{4l}{2n-1},\ f  = \frac{(2n-1)c}{4l}$

\begin{multicols}{2}
    \textit{Closed Boundary} (inverted) \\[5pt]
    $\psi_i(x_0) + \psi_r(x_0) = 0$ \hfill $\Delta \phi = \pi$ \\[5pt]
    Open end $\rightarrow$ anti-node (max $A$)

    \textit{Open Boundary} (not inverted) \\[5pt]
    $\frac{\partial \psi}{\partial x} = 0$ \hfil $\Delta \phi = 0$ \\[5pt]
    Closed end $\rightarrow$ node ($A = 0$)
\end{multicols} \vspace{-1em}

\begin{figure}[h]
    \centering
    \includegraphics[width=\columnwidth]{Figures/one_closed.jpg} \\[1em]
    \includegraphics[width=\columnwidth]{Figures/both_open.jpg}
\end{figure} \vspace{-1em}

\dotfill

\textbf{Geometric Optics}

\begin{wrapfigure}[4]{l}{0.21\columnwidth} \vspace{-1.5em}
    \includegraphics[width=0.21\columnwidth]{Figures/snell.png}
\end{wrapfigure}

Reflection: $\theta_i = \theta_r$

Snell's law: $n_1 \sin \theta_1 = n_2 \sin \theta_2$

$\sin \theta_c = \frac{n_2}{n_1},\ n_1 > n_2$ \hfill $\theta_c < \theta$ is TIL

$n = \frac{c}{v_p} = \frac{v \text{ in vacuum}}{v \text{  in medium}},\ n \geq 1$ \vspace{5pt}

Total trapping of light ocurs when material surrounded by lower $n$

Thin lens: $\frac{1}{f} = \frac{1}{u} + \frac{1}{v}$ \hfill Linear magnification: $M = \frac{I}{O} = -\frac{v}{u}$

\textbf{Sign Conventions}

$f < 0$ for diverging mirrors (convex) and lenses (concave)

$v > 0 \rightarrow$ real \hspace{10.65em} $v < 0 \rightarrow$ virtual

$I,\ M > 0 \rightarrow$ upright \hspace{7.25em} $I,\ M < 0 \rightarrow$ inverted

Mirrors: img opposite to obj $\rightarrow$ virtual

Lenses: diverging rays $\rightarrow$ virtual


% colummn break
\newpage


\begin{figure}
    \centering
    \begin{subfigure}[t]{0.45\columnwidth}
        \centering
        \includegraphics[width=\columnwidth]{Figures/convex_mirror.png}
        \caption{Convex mirror (div)}
    \end{subfigure}%
    ~
    \begin{subfigure}[t]{0.45\columnwidth}
        \centering
        \includegraphics[width=\columnwidth]{Figures/concave_mirror.png}
        \caption{Concave mirror (conv)}
    \end{subfigure}

    \begin{subfigure}[t]{0.45\columnwidth}
        \centering
        \includegraphics[width=\columnwidth]{Figures/convex_lens.png}
        \caption{Convex lens (conv)}
    \end{subfigure}%
    ~
    \begin{subfigure}[t]{0.45\columnwidth}
        \centering
        \includegraphics[width=\columnwidth]{Figures/concave_lens.png}
        \caption{Concave lens (div)}
    \end{subfigure}
\end{figure} \ \\[-5em]

\dotfill

\textbf{Optical Instruments}

Lensmaker's equation: $\frac{1}{f} = \left( \frac{n}{n_o}-1 \right) \left( \frac{1}{R_1} + \frac{1}{R_2} \right)$ \hfill $n_o \approx 1$ in air

Combined $f$ (touching): $\frac{1}{f} = \frac{1}{f_1} + \frac{1}{f_2}$

Combined $f$ (sep by $d$): $\frac{1}{f} = \frac{1}{f_1} + \frac{1}{f_2} - \frac{d}{f_1 f_2}$

$R_i > 0$ if convex, $R_i < 0$ if concave

\begin{figure}[h!]
    \centering
    \includegraphics[width=0.75\columnwidth]{Figures/mag_glass.png}
    \caption{Magnifying glass: $M_{\text{max}} = 1 + \frac{d}{f}$ \hspace{2em} $M_{\text{min}} = \frac{d}{f}$}
\end{figure}

\begin{figure}[h!]
    \centering
    \includegraphics[width=0.3\columnwidth]{Figures/microscope.png}
    \caption{Microscope: object placed just over 1 $f_o$ away. Intermediate image formed w/in 1 $f_e$. \hfill $M = m_o m_e = -\frac{L}{f_0} \left( 1 + \frac{d}{f_e} \right)$}
\end{figure}

\begin{figure}[h!]
    \centering
    \includegraphics[width=0.5\columnwidth]{Figures/telescope.png} \hfill \includegraphics[width=0.45\columnwidth]{Figures/max_diam.png}
    \caption{Telescope: $M_\theta = \frac{\phi}{\theta} = -\frac{f_0}{f_e} (= -\frac{d_0}{d_e}$ if all light captured). Here, $d_0 = $ objective diam, $d_e =$ eyepiece diam, $f_0 + f_e = L$. $L$ is distance btw eyepiece 1$^\circ$ mirror/obj lens).}
\end{figure}


% new page

\clearpage

\centering \includegraphics[width=0.55\columnwidth]{Figures/string.jpg} \flushleft \vspace{-1.5em}

\dotfill

Spherical aberrations caused by radius of curvature. \vspace{-.5em}
\begin{itemize}
    \item Diff focal lengths depending on where light hits lens/mirror.
    \item Solution: Schmidt corrector plate OR parabolic mirror.
\end{itemize}

Chromatic abberrations caused by nonuniform index of refraction. \vspace{-.5em}
\begin{itemize}
    \item Diff focal lengths depending on $\lambda$ of light.
    \item Achromats show little/no chromatic aberrations.
    \item Solution: weak concave + strong convex lens together.
\end{itemize} \vspace{-1em}

\dotfill

\textbf{Ray Tracing Principles} \vspace{-.5em}
\begin{enumerate}
    \item Ray parallel to optical axis is reflected thru $F$.
    \item Ray passing thru $F$ reflects parallel to optical axis.
    \item Ray that hits centre of mirror at optical axis reflects w/ equal and opposite angled ray.
    \item Ray that passes thru $C$ reflects along the same incident path.
\end{enumerate} \vspace{-1em}

\begin{figure}[h]
    \centering
    \begin{subfigure}[t]{0.33\columnwidth}
        \centering
        \includegraphics[height=5em]{Figures/concave_tracing.png}
    \end{subfigure}%
    ~
    \begin{subfigure}[t]{0.33\columnwidth}
        \centering
        \includegraphics[height=5em]{Figures/convex_tracing.png}
    \end{subfigure}%
    ~
    \begin{subfigure}[t]{0.33\columnwidth}
        \centering
        \includegraphics[height=5em]{Figures/conv_tracing.png}
    \end{subfigure}
\end{figure}


\end{document}
