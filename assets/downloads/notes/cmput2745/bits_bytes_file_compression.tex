\documentclass{article}
\usepackage[utf8]{inputenc}
\usepackage{multicol}
\usepackage[utf8]{inputenc}
\usepackage[T1]{fontenc}
\usepackage[margin=1in]{geometry}
\usepackage{hyperref}
\hypersetup{
    colorlinks,
    citecolor=black,
    filecolor=black,
    linkcolor=black,
    urlcolor=black
}
\usepackage{graphicx}
\usepackage{mathtools}
\usepackage{soul}
\usepackage{csvsimple}
\usepackage{hanging}
\usepackage{multicol}
\usepackage{amsmath}
\usepackage{listings}
\usepackage{amssymb}
\usepackage{float}
\usepackage{longtable}
\usepackage{pgfplotstable,filecontents}
\pgfplotsset{compat=1.9}
\usepackage[compact]{titlesec}
\titlespacing{\section}{0pt}{*0}{*0}
\titlespacing{\subsection}{0pt}{*0}{*0}
\titlespacing{\subsubsection}{0pt}{*0}{*0}
\usepackage{xcolor}
\usepackage{enumitem}
 
\definecolor{codegreen}{rgb}{0,0.6,0}
\definecolor{codegray}{rgb}{0.5,0.5,0.5}
\definecolor{codepurple}{rgb}{0.58,0,0.82}
\definecolor{backcolour}{rgb}{0.95,0.95,0.92}
 
\lstdefinestyle{mystyle}{
    backgroundcolor=\color{backcolour},   
    commentstyle=\color{codegreen},
    keywordstyle=\color{magenta},
    numberstyle=\tiny\color{codegray},
    stringstyle=\color{codepurple},
    basicstyle=\ttfamily\footnotesize,
    breakatwhitespace=false,         
    breaklines=true,                 
    captionpos=b,                    
    keepspaces=true,                 
    numbers=left,                    
    numbersep=5pt,                  
    showspaces=false,                
    showstringspaces=false,
    showtabs=false,                  
    tabsize=2
}
 
\lstset{style=mystyle}

\setlength\parindent{0pt}

\frenchspacing

\title{Python 3 Bits \& Bytes, File Compression}
\author{Eddie Guo}
\date{September 2019}


\begin{document}
\lstset{language=Python}
\maketitle

% Introduction to File Compression
\section{Introduction to File Compression}
% Topics Covered
\subsection{Topics Covered}
    \begin{multicols}{3}
        \begin{enumerate}[label=(\roman*)]
            \item Data as bits
            \item Compression
            \item Huffman coding
        \end{enumerate}
    \end{multicols}

% Bit
\subsection{Bit}
\begin{multicols}{2}
        \begin{itemize}
            \item Bit is basic unit of mem in computer
            \item 8 bits = 1 byte
            \item 4 bits = 1 nibble
            \item 2 bits = 1 crumb
        \end{itemize}

% Characters in Text Files
\subsection{Characters in Text Files}
        \begin{itemize}
            \item ASCII table defs binary rep of 256 ($2^8$) diff chars
            \item Each char has 8 bit representation: 1 byte
            \item ASCII table is w/in Unicode table
        \end{itemize}

% Always Need All Those Bits?
\subsection{Always Need All Those Bits?}
        \begin{itemize}
            \item If 1 char = 1 byte, text file w/ 1000 chars contains $\sim$1kB of data
            \item What if text file only contains chars 'a' \& 'b'?
                \begin{itemize}
                    \item Woudn't need full byte $\rightarrow$ just use single bit: 'a' = 0, 'b' = 1
                    \item Using this approach, 1000 a's \& b's would take up 1/8 space of normal ASCII rep
                \end{itemize}
        \end{itemize}

% Data Compression
\section{Data Compression}
% Compression Intro
\subsection{Compression Intro}
        \begin{itemize}
            \item \hl{Basic goal of compression: rep file using fewer bits, even if we have to store contents in unconventional format}
            \item Pros: use less mem to store file, transmit files faster
        \end{itemize}

% Another Example
\subsection{Another Example}

% Method 1
\subsubsection{Method 1}
        \begin{itemize}
            \item What if file has chars 'a', b', 'n' (ex: banana)?
            \item Can use 'a' = 00, 'b' = 01, 'c' = 10 $\rightarrow$ banana = 010010001000
            \item This gives compression rate of 1/4 that of plain-text file
        \end{itemize}

% Better Method
\subsubsection{Better Compression Choice?}
        \begin{itemize}
            \item Is there better way to compress text file w/ only 3 chars? YES!
            \item Assoc most freq character w/ 0, and remaining w/ 10 \& 11
            \item Compression rate:
                \begin{itemize}
                    \item at least 1/3 of chars go from 8 bits to 1 bit
                    \item Remaining chars go from 8 bits to 2 bits
                \end{itemize}
        \end{itemize}

% Calculate Compression Rate
\subsection{Calculate Compression Rate}
        \begin{itemize}
            \item $n_i$ = \# times 'i' occurs (where $i = a, b, c$)
            \item $n = n_a + n_b + n_n$
            \item Suppose 'a' is most freq. Then $n_a \geq n/3$
            \item Amount of bits used in compressed string is:
        \end{itemize}
    \vspace{-1.2em}
        \begin{align*}
            bits &= 1*n_a + 2*n_b + 3*n_n \\
            &= 2*n - n_a \\
            &\leq 2*n - n/3 \\
            &\leq 5/3n
        \end{align*}
        \begin{itemize}
            \item \# bytes is (5/3n)/8 = 0.2083 or less
            \item Better than 0.25n we got b4 when all 3 chars were 2 bit seq
        \end{itemize}

% Decoding
\subsection{Decoding}
    \begin{itemize}
        \item Can we decode compressed file when some chars encoded w/ 1 bit, others w/ 2 bits?
        \item Decode = 10010011
        \item 'a' = 0, 'b' = 10, 'n' = 11
        \item \includegraphics[scale=0.2]{banban.png}
    \end{itemize}

\end{multicols}

% Decoding
\section{Decoding Tree}

% Intro to Decoding Tree
\subsection{Intro to Decoding Tree}
    \begin{multicols}{2}
        \begin{itemize}
            \item If no bit seq is beginning of another in encoding, we can build decoding tree
            \item 0/1 labels on edges of root-to-leaf path = encoding of char in given leaf
            \item All chars are leaf nodes, all edges are 0/1
            \item If node not leaf, then it's an internal node
            \item Binary trees have at MOST 2 children
        \end{itemize}
    \end{multicols}

    \begin{center}
        \begin{multicols}{2}
            \includegraphics[width=0.4\textwidth]{treeStructure.png}
            \includegraphics[width=0.25\textwidth]{tree1.png}
        \end{multicols}
    \end{center}
\vspace{-2em}

% How to Use Decoding Tree
\subsection{How to Use Decoding Tree}
    \begin{multicols}{2}
        \begin{itemize}
            \item Decode bit seq using bits to traverse given tree
                \begin{enumerate}
                    \item Start at root, follow 0/1 edge according to next bit in seq
                    \item Output char at leaf when you reach one
                    \item Return to root \& repeat for next branch
                \end{enumerate}
            \item 001001101 $\rightarrow$ 00 (o) + 100 (x) + 11 (e) + 01 (n) = oxen
        \end{itemize}
        \vfill \hspace{5em} \includegraphics[scale=0.35]{oxen.png}
    \end{multicols}

% Build a Decoding Tree
\subsection{Build a Decoding Tree}
    \begin{multicols}{2}
        \begin{itemize}
            \item The key is picking encoding for each char
            \item \textbf{Req:} no bit seq for any char is the beginning (prefix) of another bit seq $\rightarrow$ this type of encoding scheme called \hl{prefix code}
            \item \textbf{Desire:} chars that occur more freq should have shorter bit seqs
            \item \textbf{Optimization problem:} construct decoding tree to minimize total \# of bits used to compress file
            \item This can be achieved using \hl{Huffman Trees} $\sim$ trees constructed according to simple \textbf{greedy procedure}
        \end{itemize}
    \end{multicols}

% Greedy Algorithms
\subsection{Greedy Algorithms}
    \begin{multicols}{2}
            \begin{itemize}
                \item At each step:
                    \begin{itemize}
                        \item \hl{Take best step we can get right now, w/o regard to eventual optimization}
                        \item \hl{Hope that by choosing local optimum at each step, you'll end up at global optimum}
                    \end{itemize}
                \item Ex: count \$6.39, using fewest bills \& coins
                    \begin{itemize}
                        \item Greedy algorithm: at each step, take largest bill/coin that does not overshoot
                    \end{itemize}
                \hspace{4em} \includegraphics[scale=0.25]{greedy.png}
            \end{itemize}
    \end{multicols}

% Huffman Coding
\section{Huffman Coding}
% Building a Huffman Tree
\subsection{Building a Huffman Tree}
    \begin{enumerate}
        \item Do freq count of all chars in file (include count of 1 for EOF sentinel)
            \begin{enumerate}
                \item Ex: freq['a'] = 10, freq['w'] = 2, freq[EOF] = 1
                \item Ultimately, keys will be bytes, not chars
                \item Total freq count of tree is sum of freqs of its leaves
            \end{enumerate}
        \item Initially, each char is single node in trivial Huffman tree
        \item While there is more than 1 tree, merge the 2 w/ the smallest freq counts
            \begin{enumerate}
                \item Make each tree a child of a new root node
                \item Doesn't matter which tree is left/right child
                \item \# on this new tree is total freq count of all leaves
                \item Repeat: pick 2 trees w/ lowest total freq count, \& merge (in case of tie, doesn't matter which one you pick)
            \end{enumerate}
        \item Merge trees T$_1$ \& T$_2$ means creating a new root node \& setting T$_1$ \& T$_2$ as its children
    \end{enumerate}

% Summary: Compress the File
\begin{multicols}{2}
\subsection{Summary: Compress the File}
    \begin{itemize}
        \item For each char, det its 0/1 compression encoding by looking at the root-to-leaf path
        \item Output the seq of 0/1 bits obtained by replacing the char w/ its compressed bits
        \item Don't forget the final seq for EOF sentinel
        \end{itemize}

% Summary: Decompress the File
\subsection{Summary: Decompress the file}
    \begin{itemize}
        \item Starting from root, traverse Huffman tree. Each bit from input seq dictates when to go L/R
        \item When you reach a leaf, output the char, return to root, continue traversing tree according to next bit(s) in seq
        \item Quit when you reach the EOF leaf
    \end{itemize}

% Why Include an EOF?
\subsection{Why Include an EOF?}
    \begin{itemize}
        \item B/c last byte of compressed file might not be "complete" (ex: 35 bits in compression seq: 4 bytes \& 3 bits, so EOF pads w/ 0); need multiple of 8 bits to transmit file
        \item $\therefore$, decoding EOF tells us when to stop
    \end{itemize}

% Considerations
\subsection{Considerations}
    \begin{itemize}
        \item To decompress a file using this approach, need to know struc of Huffman tree used to compress file in 1st place
        \item When we send compressed file, also need to send rep of Huffman tree used
    \end{itemize}

% Final Notes
\subsection{Final Notes}
    \begin{itemize}
        \item Huffman compression exploits freqs of chars $\rightarrow$ fairly simple compression scheme, but results in reduced file sizes if file contains \textbf{subset} of the 2$^8$ diff bytes (chars); \hl{no point in using Huffman tree if there are 256 diff chars in file}
        \item Huffman compression tends to work best on plain text files \& bitmap images w/ small range of colours
        \item Other compression schemes exploit other patterns, \& often target spec file types (pics, text, etc.). Some compressions allow for some data loss (ex: w/ .jpeg files)
        \item \hl{No compression can make every file smaller}
    \end{itemize}
\end{multicols}


\end{document}
