\documentclass{article}
\usepackage[utf8]{inputenc}
\usepackage{multicol}
\usepackage[margin=1in]{geometry}
\usepackage{hyperref}
\hypersetup{ colorlinks, citecolor=green, filecolor=black, linkcolor=blue, urlcolor=blue } 
\usepackage{soul}
\usepackage{csvsimple}
\usepackage{multicol}
\usepackage{amsmath}
\usepackage{listings}
\usepackage{xcolor}
\usepackage[T1]{fontenc}
\usepackage{enumitem}
 
\definecolor{codegreen}{rgb}{0,0.6,0}
\definecolor{codegray}{rgb}{0.5,0.5,0.5}
\definecolor{codepurple}{rgb}{0.58,0,0.82}
\definecolor{backcolour}{rgb}{0.95,0.95,0.92}
 
\lstdefinestyle{mystyle}{
    backgroundcolor=\color{backcolour},   
    commentstyle=\color{codegreen},
    keywordstyle=\color{magenta},
    numberstyle=\tiny\color{codegray},
    stringstyle=\color{codepurple},
    basicstyle=\ttfamily\footnotesize,
    breakatwhitespace=false,         
    breaklines=true,                 
    captionpos=b,                    
    keepspaces=true,                 
    numbers=left,                    
    numbersep=5pt,                  
    showspaces=false,                
    showstringspaces=false,
    showtabs=false,                  
    tabsize=2
}
 
\lstset{style=mystyle}

\setlength\parindent{0pt}

\frenchspacing

\title{Python 3 Classes}
\author{Eddie Guo}
\date{September 2019}


\begin{document}
\lstset{language=Python}
\maketitle

% Topics Covered
\section{Introduction to Classes}
    \begin{multicols}{2}
        \begin{enumerate}[label=(\roman*)]
            \item Procedural vs object-oriented programming
            \item Define new class
            \item Instantiate new obj
            \item Encapsulation
        \end{enumerate}
    \end{multicols}

% Procedural vs OOP
\subsection{Procedural vs Object-Oriented Programming (OOP)}
    \begin{multicols}{2}
        \begin{itemize}
            \item \textbf{Procedural programming}
                \begin{itemize}
                    \item \hl{Emphasis on actions (verb)}
                    \item Ex: \textbf{roll} dice \textit{n} times, then \textbf{build} table of data
                \end{itemize}
            \item \textbf{OOP}
                \begin{itemize}
                    \item \hl{Emphasis on objs (noun) w/ properties \& behavs}
                    \item Allows us to model real-world objs (ex: car, dog)
                \end{itemize}
        \end{itemize}
    \end{multicols}


% Example: Dogs
\subsection{Example: Dogs}
    \begin{multicols}{2}
        \begin{itemize}
            \item unique property values define each dog (ex: age, colour, size)
            \item Common behavs (ex: bark, wag tail)
        \end{itemize}
    \end{multicols}

% Python Class
\section{Python Class}
    \begin{multicols}{2}
        \begin{itemize}
            \item Class is template/blueprint
            \item Defines all attributes (properties) \& methods (behavs) that an obj will have
                \begin{itemize}
                    \item Attributes: age, size, colour
                    \item Behavs: bark, wag\_tail
                \end{itemize}
            \item Obj is \textbf{instance of a class}
                \begin{itemize}
                    \item Classes give values to all attributes of an obj
                \end{itemize}
            \item Attributes of one obj differentiates it from other objs that are instances of same class
        \end{itemize}
    \end{multicols}
\vspace{-2em}
\begin{lstlisting}
# Ex: Def New Class
# dice.py
import random

class Dice: # class name is by convention capitalized
    # self must be 1st parameter for all class defs
    def __init__(self): # method def (initializes attributes of class)
        self.sides = 6  # 'sides' is an attribute
    
    def roll(self):
        # self is ref to the obj itself (self refers to the spec die roll obj)
        return random.randint(1, self.sides) # no need to pass attribute to method inside class def
    
    def __str__(self):
        return 'Die has ' + str(self.sides) + ' sides.'

# Ex: Instantiate Obj
# use_dice.py
from dice import Dice

def play():
    # create new dice obj
    my_die = Dice() # calls __init__ method
    
    # roll my dice 3 times
    print('Roll 1:', my_die.roll()) # use dot operator on obj to invoke method
    print('Roll 2:', my_die.roll())
    print('Roll 3:', my_die.roll())
    
    # display obj
    print(my_die) # calls __str__ method

if __name__ == "__main__":
    play()

# Output
Roll 1: 3
Roll 2: 4
Roll 3: 1
Die has 6 sides.
\end{lstlisting}

% __init__()
\subsection{\texttt{\_\_init\_\_()}}
    \begin{multicols}{2}
        \begin{itemize}
            \item Special method; typically used to \textbf{initialize attributes} for new obj that's created
            \item Automatically called when an obj is instantiated (i.e., when name of class is called)
            \item Aka constructor method (not quite accurate name; see \href{https:://www.programiz.com/article/python-self-why}{here})
        \end{itemize}
    \end{multicols}

% __str__() and __repr__()
\subsection{\texttt{\_\_str\_\_()} \& \texttt{\_\_repr\_\_()}}
    \begin{multicols}{2}
        \begin{itemize}
            \item Both used to rep obj
            \item Good idea to define at least one
            \item \texttt{\_\_str\_\_} returns \textbf{informal} str rep of instance
            \item \texttt{\_\_str\_\_} is called by built-in fns \texttt{str()} \& \texttt{print()}
            \item \texttt{\_\_repr\_\_} returns \textbf{official} str rep of instance
            \item \texttt{\_\_repr\_\_} called by built-in fn \texttt{repr()}
        \end{itemize}
    \end{multicols}

% self Parameter
\subsection{\texttt{self} Parameter}
    \begin{multicols}{2}
        \begin{itemize}
            \item 1st parameter in EVERY class method
            \item Refers to obj itself
            \item Don't include as argument when invoking method of obj $\rightarrow$ self passed implicitly when using dot operator on obj
        \end{itemize}
    \end{multicols}
\vspace{-2em}
\begin{lstlisting}
# dice2.py
import random

class Dice:
    def __init__(self, howMany): # pass in additional value(s) to initialize attribute(s)
        self.sides = howMany
    
    def roll(self):
        return random.randint(1, self.sides)
    
    def __str__(self):
        return 'Die has ' + str(self.sides) + ' sides.'

# use_dice2.py
from dice2 import Dice

def play():
    # create new dice objs
    cube_die = Dice(6)
    icosahedron_die = Dice(20)
    
    # roll dice
    print('Cube roll:', cube_die.roll())
    print('Icosahedron roll:', icosahedron_die.roll())
    
    # display objs
    print(cube_die)
    print(icosahedron_die)

if __name__ = "__main__":
    play()

# Output
Cube roll: 1
Icosahdron roll: 11
Die has 6 sides.
Die has 20 sides.
\end{lstlisting}

% Encapsulation
\section{Encapsulation}
    \begin{multicols}{2}
        \begin{itemize}
            \item A class wraps up/\textbf{encapsulates} its attributes \& methods
                \begin{itemize}
                    \item Ensures all data related to an obj is contained in single struc
                \end{itemize}
            \item Attributes can be made \textbf{private} to prevent them being accessed directly by outside programs
                \begin{itemize}
                    \item Def attribute name starting w/ '\texttt{\_\_}'
                    \item Ex: \texttt{self.\_\_sides}
                \end{itemize}
            \item Implemment \textcolor{blue}{setter} \& \textcolor{red}{getter} methods to \textcolor{blue}{change} \& \textcolor{red}{access} attributes
                \begin{itemize}
                    \item Ctrl HOW attribute values can be changed \& seen
                    \item Form public interface btw program \& obj
                \end{itemize}
        \end{itemize}
    \end{multicols}

\end{document}

