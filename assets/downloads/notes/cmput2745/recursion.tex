\documentclass{article}
\usepackage[utf8]{inputenc}
\usepackage{multicol}
\usepackage[utf8]{inputenc}
\usepackage[T1]{fontenc}
\usepackage[margin=1in]{geometry}
\usepackage{hyperref}
\hypersetup{
    colorlinks,
    citecolor=black,
    filecolor=black,
    linkcolor=black,
    urlcolor=black
}
\usepackage{graphicx}
\usepackage{mathtools}
\usepackage{soul}
\usepackage{csvsimple}
\usepackage{hanging}
\usepackage{multicol}
\usepackage{amsmath}
\usepackage{listings}
\usepackage{amssymb}
\usepackage{float}
\usepackage{longtable}
\usepackage{pgfplotstable,filecontents}
\pgfplotsset{compat=1.9}
\usepackage[compact]{titlesec}
\titlespacing{\section}{0pt}{*0}{*0}
\titlespacing{\subsection}{0pt}{*0}{*0}
\titlespacing{\subsubsection}{0pt}{*0}{*0}
\usepackage{xcolor}
\usepackage{enumitem}
 
\definecolor{codegreen}{rgb}{0,0.6,0}
\definecolor{codegray}{rgb}{0.5,0.5,0.5}
\definecolor{codepurple}{rgb}{0.58,0,0.82}
\definecolor{backcolour}{rgb}{0.95,0.95,0.92}
 
\lstdefinestyle{mystyle}{
    backgroundcolor=\color{backcolour},   
    commentstyle=\color{codegreen},
    keywordstyle=\color{magenta},
    numberstyle=\tiny\color{codegray},
    stringstyle=\color{codepurple},
    basicstyle=\ttfamily\footnotesize,
    breakatwhitespace=false,         
    breaklines=true,                 
    captionpos=b,                    
    keepspaces=true,                 
    numbers=left,                    
    numbersep=5pt,                  
    showspaces=false,                
    showstringspaces=false,
    showtabs=false,                  
    tabsize=2
}
 
\lstset{style=mystyle}

\setlength\parindent{0pt}

\frenchspacing

\title{Python 3 Recursion}
\author{Eddie Guo}
\date{October 2019}


\begin{document}
\lstset{language=Python}
\maketitle

% Introduction to Recursion
\section{Introduction to Recursion}
% Topics Covered
\subsection{Topics Covered}
    \begin{multicols}{2}
        \begin{enumerate}[label=(\roman*)]
            \item What is recursion?
            \item Conditions for termination
            \item Stack frames
        \end{enumerate}
    \end{multicols}

% Recursion
\subsection{Recursion}
    \begin{itemize}
        \item Recursion occurs when fn (or method) calls itself, either directly or indirectly
        \item If problem can be resolved by solving simple part of it \& resolving rest of big problem in same way, can write a fn that solves simple part of problem then calls itself to resolve rest of problem
        \item For recursion to terminate, 2 conditions must be met:
            \begin{itemize}
                \item Must be 1/more simple cases that do NOT make recursive calls (\textbf{base case})
                \item Recursive call must somehow be simpler than original call (change state to move towards base base)
            \end{itemize}
    \end{itemize}
\vspace{-1em}
\begin{lstlisting}
def factorial(n):
    '''Return factorial of number.
    '''
    if (n == 0 or n == 1): # base case
        answer = 1
    else:
        answer = n * factorial(n-1)
    return answer
\end{lstlisting}

% Fn Activations & Frames
\subsection{Fn Activations \& Frames}
    \begin{multicols}{2}
        \begin{itemize}
            \item When fn invoked, frame/stack frame corresponding to that fn created \& pushed onto the stack
            \item Frame stores all local vars assoc w/ that fn call
            \item Frame created when fn invoked \& destroyed when fn finishes
            \item If fn invoked again, new frame is created for it w/ all its local vars
        \end{itemize}
    \end{multicols}

% Multiple Activations of a Fn
\subsection{Multiple Activations of Fn}
    \begin{multicols}{2}
        \begin{itemize}
            \item When we invoke recursive fn, fn becomes active
            \item B4 it's finished, it makes recursive call to same fn
            \item This means that when recursion used, there's >1 copy of same fn active at once
            \item Each active fn has its own frame which contains indep copies of its local vars
            \item These frames stored on the call stack
        \end{itemize}
    \end{multicols}

\end{document}
