\documentclass{article}
\usepackage[utf8]{inputenc}
\usepackage{multicol}
\usepackage[utf8]{inputenc}
\usepackage[T1]{fontenc}
\usepackage[margin=1in]{geometry}
\usepackage{hyperref}
\hypersetup{
    colorlinks,
    citecolor=black,
    filecolor=black,
    linkcolor=black,
    urlcolor=black
}
\usepackage{graphicx}
\usepackage{mathtools}
\usepackage{soul}
\usepackage{csvsimple}
\usepackage{hanging}
\usepackage{multicol}
\usepackage{amsmath}
\usepackage{listings}
\usepackage{amssymb}
\usepackage{float}
\usepackage{longtable}
\usepackage{pgfplotstable,filecontents}
\pgfplotsset{compat=1.9}
\usepackage[compact]{titlesec}
\titlespacing{\section}{0pt}{*0}{*0}
\titlespacing{\subsection}{0pt}{*0}{*0}
\titlespacing{\subsubsection}{0pt}{*0}{*0}
\usepackage{xcolor}
\usepackage{enumitem}
 
\definecolor{codegreen}{rgb}{0,0.6,0}
\definecolor{codegray}{rgb}{0.5,0.5,0.5}
\definecolor{codepurple}{rgb}{0.58,0,0.82}
\definecolor{backcolour}{rgb}{0.95,0.95,0.92}
 
\lstdefinestyle{mystyle}{
    backgroundcolor=\color{backcolour},   
    commentstyle=\color{codegreen},
    keywordstyle=\color{magenta},
    numberstyle=\tiny\color{codegray},
    stringstyle=\color{codepurple},
    basicstyle=\ttfamily\footnotesize,
    breakatwhitespace=false,         
    breaklines=true,                 
    captionpos=b,                    
    keepspaces=true,                 
    numbers=left,                    
    numbersep=5pt,                  
    showspaces=false,                
    showstringspaces=false,
    showtabs=false,                  
    tabsize=2
}
 
\lstset{style=mystyle}

\setlength\parindent{0pt}

\frenchspacing

\title{Python 3 Modules, File I/O, Dictionaries, and other bits}
\author{Eddie Guo}
\date{September 2019}


\begin{document}
\lstset{language=Python}
\maketitle

% Topics Covered
\section{Introduction}
    \begin{multicols}{2}
        \begin{enumerate}[label=(\roman*)]
            \item Map fn
            \item Import modules
            \item Reading/writing to files
            \item Dictionaries
        \end{enumerate}
    \end{multicols}

% map()
\subsection{\texttt{map()}}
    \begin{itemize}
        \item Unpacks elements of map obj \& assigns fn to indiv vars
        \item Ex: split string into list then apply map fn on list
    \end{itemize}

% Modules
\section{Modules}
% Intro to modules
\subsection{Introduction to Modules}
    \begin{multicols}{2}
        \begin{itemize}
            \item Module: file that contains Python fn defs
                \begin{itemize}
                    \item Another Python program will import module \& call fns
                    \item Allows fns to be reused
                    \item Organization: related fns grouped in one file
                \end{itemize}
            \item Modules can also contain objs that can be accessed from other files (generally constants)
            \item Rules for module names:
                \begin{itemize}
                    \item Filename should end in .py
                    \item Can't be same as Python keyword
                \end{itemize}
        \end{itemize}
    \end{multicols}
\vspace{-1em}
\begin{lstlisting}
    # use any fn in module
    import module_name
    module_name.fn1() # use dot notation when calling fn
    
    # import specific fns in module
    from module_name import fn1, fn2
    fn1() # call fn w/o ref to module
\end{lstlisting}

% __name__ == "__main__"
\subsection{\texttt{\_\_name\_\_ == "\_\_main\_\_"}}
    \begin{multicols}{2}
        \begin{itemize}
            \item Can include additional code outside fn defs in module file
                \begin{itemize}
                    \item Use \texttt{\_\_name\_\_ == "\_\_main\_\_"}
                    \item Only code in main program will run
                \end{itemize}
            \item When module file run, it's main program
            \item Imported modules are NOT in main program
        \end{itemize}
    \end{multicols}
\vspace{-1em}
\begin{lstlisting}
    # program 1
    def area(width, length):
        return width*length

    if __name__ == "__main__":
        print(area(3,5))
    15 # output 1
    
    # program 2
    import rectangle
    
    if __name__ == "__main__":
        print(rectangle.area(70,100))
    7000 # output 2; notice how it doesn't print out output 1
\end{lstlisting}

% Standard Library Modules
\subsection{Standard Library Modules}
    Useful modules:
    \vspace{-1em}
        \begin{multicols}{3}
            \begin{itemize}
                \item time
                \item sys
                \item os
                \item math
                \item random
                \item pickle
            \end{itemize}
        \end{multicols}

% math Module
\subsubsection{\texttt{math} Module}
    \begin{multicols}{2}
        \begin{itemize}
            \item Rounding fns
                \begin{itemize}
                    \item math.ceil(x), math.floor(x), math.trunc(x)
                \end{itemize}
            \item Trigonometric fns
                \begin{itemize}
                    \item math.cos(x), math.sin(x), math.tan(x)
                    \item math.acos(x), math.asin(x), math.atan(x)
                    \item math.degrees(x), math.radians(x)
                \end{itemize}
            \item Constants
                \begin{itemize}
                    \item math.pi, math.e
                \end{itemize}
        \end{itemize}
    \end{multicols}

% File I/O
\section{File I/O}
% Files for Input/Output
\subsection{Files for Input/Output}
    \begin{multicols}{2}
        \begin{itemize}
            \item So far, got user input via \texttt{input()}
            \item Some problems req lots of data, or same data to be reused
                \begin{itemize}
                    \item Manually entering data can be tedious
                    \item Instead, save data to file
                    \item Allows program to retain data btw executions
                \end{itemize}
            \item In gen, 2 types of files:
                \begin{itemize}
                    \item Binary
                    \item Text (human readable)
                \end{itemize}
        \end{itemize}
    \end{multicols}

% Using Files
\subsection{Using Files}
    \begin{enumerate}
        \item Open connection to file $\rightarrow$ create file obj
        \item Read data from file or write data to file
        \item Close connection to file (else, data may not be saved)
    \end{enumerate}

% Open File
\subsubsection{Open File}
    \begin{multicols}{2}
        \begin{itemize}
            \item Read only (default) $\rightarrow$ "r"
            \item Read \& write $\rightarrow$ "r+"
            \item Write only $\rightarrow$ "w"
            \item Append to end of file $\rightarrow$ "a"
            \item Append a "b" to above modules for binary file (ex: "rb", "t" for text (default)
        \end{itemize}
    \end{multicols}
\vspace{-1em}
\begin{lstlisting}
    student_data = open('studentData.txt', 'r') # can also specifiy path in place of 'studentData.txt'
\end{lstlisting}

% os.path: Check if file exists
\subsubsection{\texttt{os.path}: Check if File Exists}
    \begin{itemize}
        \item B4 trying to open file, may wanna check if files exists
        \item Use \texttt{os.path} module: \hl{\texttt{os.path.isfile(fname)} returns True if \texttt{fname} exists}
    \end{itemize}
\vspace{-1em}
\begin{lstlisting}
    import os.path
    
    fname = input('Enter a filename: ')
    while not os.path.isfile(fname):
        print('File does not exist')
        fname = input('Enter a filename: ')
    
    fin = open(fname, 'r')
\end{lstlisting}

% Methods to read from file
\subsubsection{Methods to Read from File}
    \begin{multicols}{2}
        \begin{enumerate}
            \item \texttt{file\_object\_name.read(size)}
                \begin{itemize}
                    \item Reads contents of file up to \texttt{size} chars (text file)
                    \item If \texttt{size} not specified, will read to end of file
                    \item Contents returned as single string, including any \texttt{\textbackslash n}
                \end{itemize}
            \item \texttt{file\_object\_name.readline()}
                \begin{itemize}
                    \item Reads single line
                    \item Line returned as string, including \texttt{\textbackslash n} if present
                \end{itemize}
            \item \texttt{file\_object\_name.readlines()}
                \begin{itemize}
                    \item Reads all lines in file
                    \item Lines returned as list of strings, including \texttt{\textbackslash n} if present
                \end{itemize}
        \end{enumerate}
    \end{multicols}
\vspace{-1em}
\begin{lstlisting}
    # Example 1: views file as list
    infile = open('names.txt', 'r')
    for line in infile:
        line = line.strip('\n')
        print(line)
    infile.close()

    # Example 2: reads file into list
    infile = open('names.txt', 'r')
    alist = infile.read().splitlines() # splitlines() splits lines at line boundaries
    # splitlines(True) includes line breaks in resulting list
    for line in alist:
        print(line)
    infile.close

    # Both examples produce identical output
\end{lstlisting}

% Writing to File
\subsubsection{Writing to File: \texttt{file\_object\_namewrite(string)}}
    \begin{itemize}
        \item Used to write data to file, or append data to file, depending on mode file was opened in
        \item \hl{Argument must be single string $\rightarrow$ use \texttt{str()} to convert (ex: if you're trying to input int, must use \texttt{file.write(str(your\_int))}}
    \end{itemize}

% Close File
\subsubsection{Close File}
    \begin{multicols}{2}
        \begin{itemize}
            \item \hl{Always close any file you open!}
                \begin{itemize}
                    \item Write: closing file flushes buffer
                    \item Read: Hogs resources if you don't close
                \end{itemize}
            \item Either use \texttt{close()} or \hl{\textbf{context manager}} to automatically close file when finished using (more Pythonic)
        \end{itemize}
    \end{multicols}
\vspace{-1em}
\begin{lstlisting}
    # Ex: context manager -> USE THIS INSTEAD OF close()
    with open('studentData.txt', 'r') as fin:
        my_data = fin.read()
\end{lstlisting}


% Dictionaries
\section{Dictionaries}
% Built-in type: Dictionary
\subsection{Built-In Type: Dictionary}
    \begin{multicols}{2}
        \begin{itemize}
            \item Dictionaries are collections of associated pairs of items
            \item Dictionaries are mutable
            \item Elements in dictionaries do NOT have order
            \item Pair consists of key \& value \{key: value\}
            \item Keys must be unique \& immutable
            \item Value can be non-unique \& mutable or immutable
        \end{itemize}
    \end{multicols}
\vspace{-1em}
\begin{lstlisting}
    cities = {
        'AB': ['Edmonton', 'Calgary'],
        'BC': ['Victoria', 'Vancouver', 'Richmond'],
        'ON': 'Toronto'
    }
\end{lstlisting}
\vspace{-1em}
    \begin{multicols}{2}
        \begin{itemize}
            \item Values accessed via keys: \texttt{cities['AB']}
            \item New pairs can be added
                \begin{itemize}
                    \item \texttt{cities['QC']='Montreal'}
                \end{itemize}
            \item Existing values can be changed:
                \begin{itemize}
                    \item \texttt{cities['QC']='Quebec'}
                \end{itemize}
            \item Existing pairs can be deleted:
                \begin{itemize}
                    \item \texttt{del cities['QC']}
                \end{itemize}
            \item \texttt{list(dict\_name)} returns list of keys of dictionary
            \item \texttt{dict\_name.keys()} returns iterable keys of dictionary
            \item \texttt{dict\_name.values()} returns iterable values of dictionary
            \item \texttt{dict\_name.items()} returns iterable pairs (key, value) of dictionary
            \item \texttt{in} returns \texttt{True} or \texttt{False} depending on whether key exists
        \end{itemize}
    \end{multicols}


\end{document}
